

\subsubsection*{Marque en la hoja de respuestas la opción acertada para las preguntas \ref{lit-1}-\ref{lit-5}}

\textbf{La primera generación de cíclopes} estaba formada por los hermanos; Arges (resplandor), Brontes (trueno) y Steropes (relámpago). Estos 3 cíclopes eran, junto a los titanes y los gigantes de las cien manos, los hijos de Gaia y Urano. Se convirtieron en los herreros forjadores del Olimpo de los Dioses dada su gran aptitud para manejar el metal. También forjaron el rayo de Zeus.\\

Urano, que odiaba a sus descendientes, mantuvo a los cíclopes presos en el interior de Gaia (la diosa Tierra) hasta que fue abatido por otro de sus hijos: Cronus (un titán). Cronus temía el poder de los inmensos cíclopes así que los volvió a encerrar. Zeus rescató a los cíclopes y éstos con sus rayos ayudaron a Zeus a vencer a los Titanes.\\

\textbf{La segunda generación de cíclopes} eran los descendientes de Poseidón y no poseían la habilidad para la metalurgia que tenían sus antecesores. Se dedicaban al pastoreo en Sicilia, donde vivían bajo ninguna ley.
El más famoso de estos cíclopes es Polifemo, uno de los protagonistas de La Odisea de Homero. En el relato se cuenta que Polifemo era especialmente cruel y consiguió atrapar a Ulises y a sus doce compañeros, a los que encerró en una cueva para devorarlos vivos. Día tras día iban cayendo miembros del grupo hasta que Ulises emborrachó con vino dulce al bobo cíclope hasta dejarlo dormido. En ese momento le atacó e hirió su único ojo. Al día siguiente, con el cíclope prácticamente ciego, consiguieron escapar camuflados bajo pieles de cabras. \begin{flushright}
{\footnotesize Tomado de:\\ http://www.seresmitologicos.net/terrestres/ciclope}
\end{flushright}
\begin{enumerate}
%%%%%%%%%%%%%%%%%%%%%%%%%%%%%%%%%%

\item Según el texto anterior, el ciclope es un personaje común en la literatura \label{lit-1}


\begin{enumerate}[(A)]
\item Oriental
\item Americana
\item Griega
\item Trágica 
\end{enumerate}


%%%%%%%%%%%%%%%%%%%%%%%%%%%%%%%%%%

\item Según el texto Ulises escapa del ciclope mediante:\label{lit-2}

\begin{enumerate}[(A)]
\item La argucia.
\item La discusión. 
\item La verdad. 
\item El dialogo.
\end{enumerate}

%%%%%%%%%%%%%%%%%%%%%%%%%%%%%5

\item según el último párrafo del texto, es posible inferir que los ciclopes son seres que para Ulises representan: \label{lit-3}
\begin{enumerate}[(A)]
\item Ayuda. 
\item Oposición.
\item Guía. 
\item Proveedores.
\end{enumerate}

%%%%%%%%%%%%%%%%%%%%%%%%%%%%%
\item Según el texto, desde su origen, los ciclopes han generado\label{lit-4}
\begin{enumerate}[(A)]
\item Conflicto.
\item Alegrías.
\item Temor.
\item Guerras.
\end{enumerate}


%%%%%%%%%%%%%%%%%%%
\item a partir del texto se puede inferir que los ciclopes  son:\label{lit-5}

\begin{enumerate}[(A)]
\item Seres mitológicos que vivieron en Sicilia.
\item Seres fabulosos creados por la humanidad. 
\item Los protagonistas de la odisea
\item Los antecesores de los dioses
\end{enumerate}



\subsubsection*{Preguntas de selección múltiple con única respuesta.  Marque en la hoja de respuestas la opción acertada para las preguntas \ref{lit-6}-\ref{lit-14} a partir del siguiente texto}

\begin{center}
\textbf{LA CONCUPISCENCIA DE LOS SENTIMIENTOS\\
(Texto argumentativo)}
\end{center}


Uno de los rasgos más sobresalientes en la televisión durante los últimos años es la \textbf{\underline{exhibición impúdica}} de los sentimientos como recurso \textbf{\underline{infalible}} para el incremento de las audiencias. Se ha comprobado que la utilización \textbf{\underline{demagógica }} del dolor ajeno vende, y se ha explotado tanto, en los informativos como, en los \textit{reality shows}. En la mayor parte de los casos no se pretende analizar las situaciones de dolor, añadiendo racionalidad a la emotividad, sino embotar las sensibilidades y las conciencias anulando toda racionalidad y convirtiendo la lagrima en espectáculo.

La hipertrofia del sentimiento se corresponde con la represión de la racionalidad.  Los problemas se banalizan, se trivializan. No se pretende proyectar algo de luz sobre las situaciones dolorosas, sino aprovecharse comercialmente de ellas. Y no sólo se exhiben impúdicamente las emociones, sino que se recurre también a la humillación pública, sometiendo a concursantes y a participantes de reality shows a pruebas denigrantes.

En Estados Unidos esta tendencia alcanza límites delirantes; por ejemplo, al transmitir en vivo juicios reales sobre los casos más morbosos, o al transmitir en directo una ejecución, conseguido el \textbf{\underline{beneplácito}} del juez federal. La cadena estadounidense Court TV, que comenzó a emitir en julio de 1991, se dedicaba a retransmitir juicios reales las 24 horas del día.  El record de audiencia de la cadena lo tiene la retransmisión del juicio de Lyle y Eric Menéndez, dos hermanos de Beverly Hills que asesinaron a sus padres en el verano del 89. La prensa aireó el caso de un expolicía de Nueva York, Stanley Orlen, que atrasó su paso por el quirófano porque no quería perderse ni un solo día el juicio de los hermanos Menéndez. El lema de la emisora es elocuente. ``\textit{Si Court TV creara un poco más de adicción, sería ilegal}''

En Estado unidos unos 45 millones de personas siguen cada día los 12 \textit{talk shows}  que emiten las cadenas más populares del país. Seguramente estos espectadores esperan encontrar más confortables sus vidas al compararlas con las miserias ajenas. En marzo de 1995 un hombre mataba a otro en casa porque lo había humillado al declararle su amor ante las cámaras en uno de esos \textit{talk shows} matutinos.

En Italia, en 1994, la RAI-3 ideó y comenzó a emitir con éxito el programa: ``\textit{El, Ella y el Otro}'', emisión dedicada a parejas totas por la aparición de un tercero, amante heterosexual u homosexual de uno de ellos. Se realizaba con la presencia en el estudio de los tres interesados, que exhibían, a veces, a voz en grito, sus problemas, formulaban públicamente sus acusaciones, confesaban sus traumas… 

También las telenovelas forman parte de este resurgir de la pornografía de los sentimientos en una sociedad que, curiosamente, reprime sus sentimientos en la mayor parte de los ámbitos de la vida cotidian.  La Asociación de Telespectadores y Radioyentes hablaba de que la vida se ha dramatizado y ``ya sólo se llora ante el aparato de televisión''.

La pornografía de los sentimientos pone de manifiesto un extraordinario sentido de \textbf{\underline{exhibicionismo}} por parte de algunos ciudadanos y, además, una curiosidad morbosa cercana al voyerismo enfermizo, por parte de los espectadores. Violencia, sexo, mal gusto copan a menudo las pantallas.  No es de extrañar que en 1992 Gabe Pressman, reportero de la NBC exclamara: ``\textit{Emitimos una tonelada de basura al día}''.

Seguramente si la basura seduce, es porque remite inconscientemente al espectador a las dimensiones más oscuras de sí mismo, porque da cuerpo narcisísticamente a su fascinación por el mal,  por el dolor, por  la destrucción y la muerte, porque actúa como espejo inconsciente de las zonas más turbias del propio siquismo.


\begin{flushright}
{\footnotesize Ferres, Joan. 206. Televisión subliminal. Barcelona.  Paid\'os.}
\end{flushright}




%%%%%%%%%%%%%%%%%%%%%%%%%%%%%
\item  Las palabras exhibición impúdica corresponden respectivamente a un: \label{lit-6}

\begin{enumerate}[(A)]
\item Verbo y adverbio.
\item Sustantivo y adjetivo.
\item Artículo y preposición.
\item Conjunción y sustantivo.
\end{enumerate}

%%%%%%%%%%%%%%%%%%%%%%%%%%%%%
\item  En la frase... ``en la mayor parte de los ámbitos de la vida cotidiana'' las palabras subrayadas son respectivamente:\label{lit-7}

\begin{enumerate}[(A)]
\item Sustantivo - adjetivo.
\item Conjunción - pronombre.
\item Determinante - preposición.
\item Adverbio- verbo.
\end{enumerate}
%%%%%%%%%%%%%%%%%%%%%%%%%%%%%
\item En el primer párrafo del texto, las palabras subrayadas se pueden reemplazar respectivamente por: \label{lit-8}

\begin{enumerate}[(A)]
\item Muestra - vergonzosa - exhibicionista - habladora. 
\item Presentación - banal - charlatana - educativa- 
\item Exposición - correcta - precipitado- desahogado.
\item Manifestación - escabrosa - verdadero - populista.
\end{enumerate}
%%%%%%%%%%%%%%%%%%%%%%%%%%%%%
\item La palabra beneplácito, utilizada en el tercer párrafo, hace referencia a: \label{lit-9}


\begin{enumerate}[(A)]
\item Consentimiento.
\item Venerable.
\item Afecto.
\item Ejemplo. 
\end{enumerate}
%%%%%%%%%%%%%%%%%%%%%%%%%%%%%
\item Un sinónimo de la palabra exhibicionismo en el último párrafo, puede ser: \label{lit-10}

\begin{enumerate}[(A)]
\item Emboscada.
\item Exteriorización.
\item Peregrinación.
\item Exclusivismo. 
\end{enumerate}
%%%%%%%%%%%%%%%%%%%%%%%%%%%%%
\item El autor califica de voyeristas a los televidentes porque: \label{lit-11}


\begin{enumerate}[(A)]
\item Sienten satisfacción personal al observar las emociones y sentimientos más íntimos en personas de la televisión.
\item Manifiestan un tipo de parafilia relativa al disfrute ocasionado al observar actos íntimos.
\item Demuestran una evidente preferencia por los viajes en lugar de ver televisión.
\item Él siente un gran desprecio por los personajes de la televisión y por los televidentes.
\end{enumerate}
%%%%%%%%%%%%%%%%%%%%%%%%%%%%%
\item Para el autor, la televisión cautiva a los espectadores en gran medida. Porque: \label{lit-12}

\begin{enumerate}[(A)]
\item Es un medio de comunicación masivo al cual todo el mundo tiene acceso.
\item La televisión en Estados Unidos es de bastante contenido y bien hecha.
\item La televisión y el Internet atrofian el cerebro de los jóvenes.
\item Sienten más confortables sus vidas al compararlas con las tragedias ajenas. 
\end{enumerate}
%%%%%%%%%%%%%%%%%%%%%%%%%%%%%
\item A partir del texto, se puede inferir, que la posición del autor frente a la televisión es: \label{lit-13}

\begin{enumerate}[(A)]
\item Aprobatoria.
\item Crítica.
\item Desmedida.
\item Legal.
\end{enumerate}
%%%%%%%%%%%%%%%%%%%%%%%%%%%%%
\item En el texto, Gabe Pressman afirma: ``Emitimos una tonelada de basura al día''. Dicha expresión significa que: \label{lit-14}


\begin{enumerate}[(A)]
\item La televisión está produciendo programas de alta calidad y contenido.
\item Las programadoras desconocen los planes de reciclaje.
\item La televisión está produciendo programas que carecen de alta calidad y contenido.
\item La televisión debe apoyar el programa de ``Basura Cero''.
\end{enumerate}



%%%%%%%%%%%%%%%%%%%%%%%%%%%%%%%%%%%%%5

\subsubsection*{Lea los siguientes párrafos y responda las preguntas \ref{lit-15}-\ref{lit-20}}

\begin{center}
\textbf{¿CUÁL SERÁ EL FIN DE LA TIERRA?	}
\end{center}

 
En estas condiciones, también la Tierra se iría enfriando lentamente. El agua se congelaría y las regiones polares serían cada vez más extensas. En último término, ni siquiera las regiones ecuatoriales tendrían suficiente calor para mantener la vida.  El océano entero se congelaría en un bloque macizo de hielo, e incluso el aire se licuaría primero y se congelaría luego. Durante billones de años, esta Tierra gélida (y los demás planetas) seguiría girando alrededor del difunto Sol. Pero aun en esas condiciones, la Tierra, como planeta, seguiría existiendo.


 En tales condiciones, es probable que la Tierra se convierta en un ascua y luego se vaporice. En ese momento, la Tierra, como cuerpo planetario sólido, acabará sus días. Pero no os preocupéis demasiado: échale todavía ocho mil millones de años.


Hasta los años treinta, parecía evidente que el Sol, como cualquier otro cuerpo caliente, tenía que acabar enfriándose. Vertía y vertía energía al espacio, por lo cual este inmenso torrente tendría que disminuir y reducirse poco a poco a un simple chorrito. El Sol se haría naranja, luego rojo, iría apagándose cada vez más y, finalmente, se apagaría.	


 Sin embargo, durante la década de los treinta, los científicos nucleares empezaron a calcular por primera vez las reacciones nucleares que tienen lugar en el interior del Sol y otras estrellas. Y hallaron que, aunque el Sol tiene que acabar por enfriarse, habrá períodos de fuerte calentamiento antes de ese fin. Una vez consumida la mayor parte del combustible básico, que es el hidrógeno, empezarán a desarrollarse otras reacciones nucleares que calentarán el Sol y harán que se expanda enormemente. Aunque emitirá una cantidad mayor de calor, a cada porción de su ahora vastísima superficie le tocará una fracción mucho más pequeña de ese calor y será, por tanto, más fría.  El Sol se convertirá en una masa gigante roja. 
Isaac Asimov, Cien preguntas básicas sobre la ciencia.  

\begin{flushright}
{\footnotesize Tomado de:\\
Isaac Asimov, Cien preguntas básicas sobre la ciencia.\\
 http://www.xtea.cat/~jgenover/ordtexto3.htm}
\end{flushright}



%%%%%%%%%%%%%%%%%%%%%%%%%%%%%%%%555
\item La opción que corresponde al orden lógico del texto es: \label{lit-15}


Final del formulario

\begin{enumerate}[(A)]
\item  A - B - C - D 
\item  C - A - D - B
\item  C - D - B - A
\item  B - C -  D -  A 
\end{enumerate}
%%%%%%%%%%%%%%%%%%%%%%%%%%%%%%%%555
\item La idea principal y la conclusión del texto se encuentran respectivamente en los párrafos: \label{lit-16}


\begin{enumerate}[(A)]
\item  A - D
\item  B - C 
\item  A - C
\item  C - B 
\end{enumerate}
%%%%%%%%%%%%%%%%%%%%%%%%%%%%%%%%555
\item  Según su estructura el texto completo puede ser clasificado como:\label{lit-17}


\begin{enumerate}[(A)]
\item  Deductivo porque presenta una idea general y cierra con una idea particular.
\item  Inductivo porque parte de una idea especifica.
\item  Deductivo-inductivo porque la idea general se desarrolla con ideas secundarias y cierra con una conclusión.
\item  Inductivo-deductivo la idea principal está en la mitad del texto, y las ideas de la conclusión se deben desarrollar.
\end{enumerate}
%%%%%%%%%%%%%%%%%%%%%%%%%%%%%%%%555
\item  Según su finalidad el párrafo A del texto anterior se puede clasificar como:\label{lit-18}

\begin{enumerate}[(A)]
\item  Expositivo-Argumentativo porque presenta una idea y la sustenta con argumentos.
\item   Narrativo-Argumentativo porque presenta una idea y la sustenta con argumentos que relatan una historia. 
\item  Descriptivo- Expositivo porque presenta una idea y las características de un hecho, objeto o fenómeno.
\item  Argumentativo-Descriptivo porque presenta una idea y las características de un hecho, objeto o fenómeno.
\end{enumerate}
%%%%%%%%%%%%%%%%%%%%%%%%%%%%%%%%555
\item Según su estructura el párrafo D del texto anterior se puede clasificar como:\label{lit-19}

\begin{enumerate}[(A)]
\item  Conclusión porque cierra las ideas.
\item  Introducción porque contiene la idea principal.
\item  Enlace porque conecta información presentada anteriormente.
\item  Desarrollo porque presenta argumentos que sustentan la idea principal.
\end{enumerate}
%%%%%%%%%%%%%%%%%%%%%%%%%%%%%%%%555
\item  La expresión Sin embargo corresponde a un conector formado por una conjunción de tipo: \label{lit-20}


\begin{enumerate}[(A)]
\item  Aditivo.
\item  Temporal.
\item  Locativo.
\item  Adversativo.
\end{enumerate}

\subsubsection*{\textbf{PREGUNTAS ABIERTAS}
En relación con el siguiente texto responda las preguntas \ref{lit-21}-\ref{lit-25}}

A manera de introducción se hace necesario definir la literatura desde su dimensión filosófica ya que ésta valida el carácter epistemológico de la misma. La literatura es una episteme porque es una forma de conocer el mundo y la manera como nos relacionamos con él. La obra literaria permite reflexionar sobre la vida, el ser, la sociedad, la historia, la cultura, e incluso sobre el conflicto mismo de ser humano. Sin embargo, el objeto de conocimiento de la literatura no se enmarca dentro de las ciencias positivas,  caracterizadas por su racionalidad y empirismo, así como tampoco responde a las lógicas del mundo cotidiano. El objeto de conocimiento de la literatura permite construir un saber desde la sensibilidad  porque el elemento central  de su existir es la correlación entre el hombre y su mundo de vida, es decir desde la fenomenología.


 La fenomenología, tal como la define Gadamer en su texto “Verdad y Método” se constituye como la ciencia que orienta  y valida el saber del hombre sobre sí mismo en el ámbito de la experiencia, esto permite concebir la literatura como  recurso que permite al hombre construir un saber de si en el mundo de la vida. 
 
 
Sartre se basa en la fenomenología para definir la literatura porque la concibe como una cuestión esencial de la cultura en tanto es un arte y como tal se convierte en un componente fundamental de la condición del ser. En ese sentido la literatura según Sartre se transforma en un símbolo del ser al darle sentido a su proceso existencial histórico y cultural a través del lenguaje.


 Para sustentar la definición de Jean-Paul Sartre en el presente documento se sintetizan tres premisas fundamentales que se inter-relacionan y construyen mutuamente alrededor de la literatura, la primera es la escritura como arte, la segunda es la escritura como emancipación, y la tercera la dialéctica entre escritor, lector y obra. Además, se establece la relación entre la literatura y la labor docente.
 
 
En primer lugar, la escritura es entendida como arte porque el escritor pinta el mundo con palabras, en la poética se concibe el lenguaje como una metáfora del mundo que permite la metamorfosis de las emociones y sentimientos en las palabras, al punto que es imposible para el escritor diferenciar los unos de las otras. El hombre se construye a través del lenguaje porque, como dice Chomsky, estamos hechos desde y para el lenguaje. Dependiendo del uso que el escritor le dé al lenguaje se le reconocerá como prosista o como poeta. El primero se rodea de las palabras y las concibe como un espejo del mundo y por lo tanto Sartre lo niega como parte de la literatura. Dado que para él la literatura no es un marco de referencia y mucho menos existe para hacer una copia de las experiencias es imposible establecer una realidad universal.


Por otro lado, para el poeta las palabras son ``el medio para atrapar la realidad huidiza''; por lo tanto, dan cuenta de la forma que el hombre percibe el mundo. El poeta tiene el poder de crear mundos, es decir, la poesía le da la posibilidad de crear un conjunto infinito de posibilidades de sentido. Y entonces aparece la cuestión ¿cuál es el compromiso del escritor?  Al respecto, Sartre responde que el prosista es más comprometido que el poeta porque tiene la posibilidad de servirse de las palabras para crear conciencia. El prosista comprometido debe revelar el mundo y provocar indignación; de este modo, logrará establecer una relación con el lector y llegar a un acuerdo que les permitirá transformar la realidad. Por su parte, el poeta comprometido debe trascender su realidad histórica y escribir por y para la libertad.


Ese compromiso con la libertad humana define la escritura como emancipación al reflexionar el rol del escritor y del lector desde el ámbito social. En el cual se requiere de la negociación entre el uno y el otro. El poeta crea para revelar y producir nuevos mundos bajo sus propias reglas, esto significa que dada su función social de `guardián de los valores ideales', escribe para la libertad, ésa es su única manera de existir, para Sartre `el poeta no es libre por decisión es libre de hecho'. Sin embargo, para que ese ideal de libertad sea alcanzado se requiere de la relación dialógica y simbiótica entre el escritor, la obra y el lector. Esto quiere decir que el escritor necesita de un lector comprometido, dispuesto a construir el significado de la obra que él proyecta. La libertad del lector le permite descubrir lo bello que tiene el mundo desde su subjetividad. Entre tanto, La obra por sí misma no existe, pasa a ser un ente  dependiente de `su' lector-creador porque gracias a él la obra pasa de existir a ser. Así pues, el escritor aprende a confiar en la capacidad creadora del lector apelando a su libertad y el lector cumple su función y re-crea el mundo que le es revelado a través de la experiencia estética. 


La relación dialéctica del triado escritor, obra, lector nos lleva a la tercera premisa la cual invita a relacionar estos tres elementos dentro de un contexto histórico para lograr una sociedad consciente de sí misma. Desde  un punto de vista fenomenológico esa relación puede ser descrita de la siguiente manera. El escritor se erige como mediador y revelador del conflicto entre la organización social  y el lector, luego es el lector quien establece la esencia de la obra, un lector ideal comprometido con la re-creación de la obra y con la toma de conciencia sobre el uso de su libertad como fin último será capaz de dar sentido a la existencia humana desde la obra.


 Así pues, la obra se establece como un ente inacabado, fuente de conocimiento, que permite relacionar experiencias, comunicar deseos, liberar lo humano en tanto se construye en cada lector y trasciende a una realidad metafísica. 

\item ¿Qué es literatura? \label{lit-21}

\item ¿Por qué la literatura trasciende su tiempo y aún no ha llegado a su fin?\label{lit-22}
	
\item ¿Cuál es la relación entre filosofía y literatura?\label{lit-23}

\item ¿Cuál debe ser el compromiso del autor y de lector?\label{lit-24}

\item ¿La literatura tiene carácter de conocimiento?\label{lit-25}
	

%%%%%%%%%%%%%%%%%%%%%%%%%%%%%
\end{enumerate}
%%%%%%%%%%%%%%%%%%%%%%%%%%%%%5


