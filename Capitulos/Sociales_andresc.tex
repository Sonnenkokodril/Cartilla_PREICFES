

\subsubsection*{Con base en el siguiente texto responda las preguntas: \ref{socandres-1} a \ref{socandres-3}}
Hasta el siglo XV América fue un continente habitado mayoritariamente por indígenas de diversas culturas: Mayas, Aztecas, Chibchas, Incas, etc. No obstante, se dice en algunos libros de historia que en 1492 surge el llamado ``descubrimiento de América''. El diccionario de la Real Academia Española define ``descubrimiento'' como: ``Hallazgo, encuentro, manifestación de lo que estaba oculto o secreto o era desconocido''.

\begin{enumerate}
\item Lo anterior presupone una paradoja porque: \label{socandres-1}

\begin{enumerate}[(A)]
\item 1492 es un año que está inserto dentro del siglo XV.
\item La definición de la Real Academia Española no contempla hechos históricos.
\item No se puede descubrir algo que ya ha sido descubierto por otros.
\item No se puede determinar exactamente quién descubrió América.
\end{enumerate}

\item Una explicación acertada para este fenómeno es\label{socandres-2}
\begin{enumerate}[(A)]
\item La fuerza de la costumbre ha configurado el contenido de algunos libros de historia.
\item Colón pensaba que llegaba a las Indias descubriendo accidentalmente América.
\item Los recursos informativos en el siglo XV eran escasos y por ende no pudieron hacer un rastreo histórico exacto.
\item Dichos libros de historia manejan una perspectiva Eurocéntrica.
\end{enumerate}

\item Justifique su respuesta \label{socandres-3} \hrulefill\\
\_\hrulefill\\
\_\hrulefill\\
\_\hrulefill.

\subsubsection{Responda las preguntas de la \ref{socandres-4} a la \ref{socandres-6} con base en el siguiente texto.}
``A los cincuenta años \rule{3cm}{1pt} interviene por primera vez en la historia de la inquisición española. <<Mi nombre es Liberación>>, le hace exclamar Víctor Hugo. La realidad es bastante diferente y más desgraciada, atribuyéndosele la responsabilidad de muchas muertes de herejes.''\\{\footnotesize Tomado de: De Juan, J. y Pérez, F. (1985) La inquisición. Editora Cinco S.A. Bogotá. Pág. 35.}

\item El anterior texto hace referencia a un personaje histórico conocido como el primer gran inquisidor o primer inquisidor general, el cual es: \label{socandres-4}

\begin{enumerate}[(A)]
\item Fray José de Ascanio, nacido en 1210.
\item Papa Pío VI (Ángelo Onofrio), nacido en 1717.
\item Fray Tomás de Torquemada, nacido en 1420.
\item Papa Rodrigo Borgia (Alejandro VI), nacido en 1431. 
\end{enumerate}

\item El principal objetivo de la inquisición fue:\label{socandres-5}
\begin{enumerate}[(A)]
\item Torturas selectivas a los desertores de la Santa Iglesia Católica.
\item Supresión de la herejía, principalmente en el seno de la Santa Iglesia Católica.
\item Luchas cruzadas en contra de fundamentalistas de otras religiones.
\item Acaparamiento del oro proveniente de la recién descubierta América, por parte de la Santa Iglesia Católica. 
\end{enumerate}

\item La Inquisición es una institución que históricamente ha despertado fuertes críticas en diversos sectores debido a:\label{socandres-6}
\begin{enumerate}[(A)]
\item La ausencia de radicalismo en los procedimientos para la supresión de la herejía.
\item El acaparamiento de las riquezas de pueblos periféricos en torno a la figura del Papa.
\item Extralimitaciones en los procedimientos para combatir la herejía como el asesinato, la tortura y otros crímenes.
\item La falta de tolerancia ante las ideas religiosas profesadas por las comunidades musulmanas y asiáticas.
\end{enumerate}

\subsubsection*{Responda las preguntas de la \ref{socandres-7} a la \ref{socandres-10} con base en el siguiente texto}
``La condición esencial de la existencia y de la dominación de la clase burguesa es la acumulación de la riqueza en manos de particulares, la formación y el acrecentamiento del capital. La condición de existencia del capital es el trabajo asalariado. El trabajo asalariado descansa exclusivamente sobre la competencia de los obreros entre sí. El progreso de la industria, del que la burguesía, incapaz de oponérsele, es agente involuntario, sustituye el aislamiento de los obreros, resultante de la competencia, por su unión revolucionaria mediante la asociación. Así, el desarrollo de la gran industria socava bajo los pies de la burguesía las bases sobre las que ésta produce y se apropia lo producido. La burguesía produce, ante todo, sus propios sepultureros.''\\
\begin{footnotesize}
Tomado de: Marx Karl, Engels F. (1970) Manifiesto del partido comunista y otros escritos políticos. Editorial Grijalbo. México, D.F. Pág. 38.
\end{footnotesize}

\item Con la frase: “La condición esencial de la existencia y de la dominación de la clase burguesa es la acumulación de la riqueza en manos de particulares” Marx se refiere a:\label{socandres-7}

\begin{enumerate}[(A)]
\item Que es requisito para la constitución de una sociedad burguesa el atesoramiento de las riquezas por parte de una clase social determinada.
\item Que sólo a través de la apropiación y acumulación de la riqueza por parte de la clase burguesa puede ésta existir y mantener su dominio.
\item Que la burguesía para mantener su existencia y dominio debe impedir la existencia de una esfera pública social determinada.
\item Que la burguesía constituye en sí misma un modo de producción determinado puesto que el atesoramiento de la riqueza se da únicamente en su respectiva clase social.
\end{enumerate}

\item De la frase: “El trabajo asalariado descansa exclusivamente sobre la competencia de los obreros entre sí”, se podría concluir que:\label{socandres-8}

\begin{enumerate}[(A)]

\item Los obreros deben competir entre ellos mismos en la producción de bienes puesto que de ello dependerá el monto y renta de su salario.
\item Los obreros deben competir entre ellos mismos para asegurar un lugar dentro del sistema de producción y con ello un salario y su subsistencia.
\item Los obreros deben competir para obtener un lugar dentro del ejército de reserva industrial y de esta manera asegurar su subsistencia.
\item Los obreros deben competir entre sí en la rapidez mecánica de su ejercicio laboral para dar paso a nuevas formas de producción.
\end{enumerate}

\item Cuando Marx afirma que: “El progreso de la industria, del que la burguesía, incapaz de oponérsele, es agente involuntario, sustituye el aislamiento de los obreros, resultante de la competencia, por su unión revolucionaria mediante la asociación.” Se puede inferir que:\label{socandres-9}

\begin{enumerate}[(A)]
\item Con el necesario progreso de la industria, es inevitable que los obreros se aglutinen en fábricas y de esta manera, al permanecer juntos e identificar una explotación conjunta -como clase proletaria- se asocien con el fin de transformar sus condiciones materiales de vida a través de una revolución que derroque a la burguesía.
\item La burguesía, al ser incapaz de frenar los avances de la industria, es así mismo incapaz de frenar la competencia que sostienen los obreros entre sí, competencia que dará lugar a la superación asociativa de los trabajadores y con ello, al establecimiento de una nueva clase burguesa que derroque a la anterior.
\item La unión revolucionaria de los trabajadores al derrocar a los señores feudales y con ello, al viejo sistema de producción, se erige como nuevo ente social dominante cuyo fin es la superación del aislamiento industrial y de esta manera eliminar el sistema de competencia para establecer el nuevo modo de producción socialista.
\item Al agudizarse los avances de la industria, los obreros se unen en las diversas fábricas identificándose ideológicamente entre sí, y conformando la clase obrera -o proletariado-, que a través de la competencia establecerá una asociación revolucionaria cuyo objetivo es el derrocamiento de la clase burguesa.     
\end{enumerate}

\item Según los planteamientos de Karl Marx, el sistema capitalista es:\label{socandres-10}

\begin{enumerate}[(A)]
\item Justo, porque toda persona recibe una retribución o salario que se establece conforme al trabajo realizado.
\item Injusto, porque se sostiene sobre la explotación de los trabajadores a raíz, principalmente, de la apropiación de la plusvalía por parte de la burguesía.
\item Justo, porque el ser burgués o proletario no es algo inmanente a la naturaleza humana, sino que dependerá del esfuerzo de cada sujeto.
\item Injusto, porque no se combate la desigualdad inherente a la naturaleza humana.
\end{enumerate}

``Todas las luchas que se libran dentro del Estado, la lucha entre la democracia, la aristocracia y la monarquía, la lucha por el derecho de sufragio, etc., no son sino las formas ilusorias bajo las que se ventilan las luchas reales entre las diversas clases''\\
{\footnotesize Tomado de: Marx C, Engels F. (1976) La ideología Alemana. Ediciones Calarcá. Bogotá. Pág. 24.}
\item Teniendo en cuenta la categoría “Lucha de clases”, bajo el sistema moderno de producción, las clases sociales identificadas por Marx y Engels que se enfrentan son:\label{socandres-11} \hrulefill\\
\_\hrulefill\\
\_\hrulefill\\
\_\hrulefill.

\subsubsection*{Responda a las preguntas \ref{socandres-12} y \ref{socandres-13} con base en el siguiente texto}

``Mijail Gorbachov escribe en su libro ``La Perestroika'' lo siguiente: ``Por supuesto que la Perestroika ha sido ampliamente estimulada por nuestro descontento por la manera en que han funcionado las cosas en nuestro país en los años recientes. Pero en mucha mayor medida fue impulsada por la conciencia de que el potencial del socialismo había sido poco utilizado.''\\
{\footnotesize Tomado de: Gorbachov, M. (1988) La perestroika. Editorial Oveja Negra. Bogotá. Pág. 07.}

\item El significado de la palabra ``Perestroika'' es: \label{socandres-12} \hrulefill\\
\_\hrulefill\\
\_\hrulefill\\
\_\hrulefill.

\item Una de las principales consecuencias de la aplicación de La Perestroika fue: \label{socandres-13}

\begin{enumerate}[(A)]
\item Fin del culto a la personalidad en Rusia.
\item Fin del socialismo soviético y disolución de la URSS.
\item Ascenso de Vladimir Putin al poder.
\item Inicio de la Guerra fría entre la URSS y EEUU.
\end{enumerate}


\item El orden correcto para los siguientes acontecimientos que configuran la historia mundial es: \label{socandres-14}


\begin{enumerate}[1.]
\item Revolución Francesa
\item Caída del muro de Berlín 
\item Invasión de América
\item Primera guerra mundial
\item La perestroika
\item Revolución de Octubre
\item Segunda guerra mundial
\item Guerra fría
\end{enumerate}

\begin{enumerate}[(A)]
\item 3, 1, 6, 4, 7, 8, 5 y 2
\item 3, 1, 4, 6, 7, 8, 2 y 5
\item 3, 1, 6, 4, 5, 7, 2 y 8
\item 3, 1, 4, 6, 7, 8, 5 y 2  
\end{enumerate}
\item El Apartheid fue un sistema de segregación racial que tuvo lugar principalmente en Sudáfrica. Dicho sistema consistía básicamente en la creación de lugares separados para los diferentes grupos raciales, la no autorización de matrimonio o contacto sexual entre blancos y negros, además de la prohibición de que los negros ejercieran el voto con el fin de que la minoría blanca mantuviese el poder político. Es pues, una de las mayores muestras históricas de discriminación que ha existido. \label{socandres-15}

Si el reglamento de un colegio estatal plantea que sólo las personas con orientación heterosexual tienen derecho a ingresar en ella. ¿Se puede considerar esta situación, moralmente como análoga al Apartheid?, ¿por qué?
\hrulefill\\
\_\hrulefill\\
\_\hrulefill\\
\_\hrulefill.

\item El 1 de Enero de 1994 se da a conocer, a través de un levantamiento armado, en Chiapas-México el grupo guerrillero autodenominado Ejército Zapatista de Liberación Nacional (EZLN). Las exigencias de este grupo insurgente han girado en torno a una transformación radical de la sociedad cuyas bases se construyan teniendo como ejes fundamentales la democracia, la libertad y la justicia. Desde la fecha de su aparición y hasta el día de hoy algo que ha asombrado a diversos sectores sociales, académicos y políticos es que el EZLN no se plantea como objetivo la toma del poder político. \label{socandres-16}

Este asombro se debe a:

\begin{enumerate}[(A)]
\item Que es desacertada la pretensión de lograr objetivos tales como la democracia, la libertad y la justicia a través de un levantamiento armado, pues habría una falta de coherencia entre medios y fines.
\item Que no se explica cómo lograr una transformación radical de la sociedad descartando la toma del poder político por parte de grupos armados al margen de la ley.
\item Que hay una ruptura del paradigma ideológico guerrillero en tanto que, por lo general, los diferentes grupos guerrilleros que emergieron a lo largo del siglo XX tenían como requisito indispensable para el logro de una transformación social radical, la toma del poder político.
\item Que el EZLN no desea, sea cual sea la forma de lucha, una toma real y efectiva del poder político vigente.
\end{enumerate}

\item El  orden correcto para los siguientes hechos que configuran la historia de Colombia es:\label{socandres-17}

\begin{enumerate}[1.]
\item Asesinato de Jorge Eliécer Gaitán.
\item Guerra de los mil días.
\item Frente Nacional.
\item Política de seguridad democrática.
\end{enumerate}
\begin{enumerate}[(A)]
\item 1, 2, 3 y 4
\item 2, 1, 3 y 4
\item 1, 3, 2 y 4
\item 2, 1, 4 y 3
\end{enumerate}
\item Con el surgimiento de la Constitución Política de Colombia de 1991 nacen así mismo diversos mecanismos jurídicos para salvaguardar y hacer efectivos los derechos de los ciudadanos. Uno de estos mecanismos es la acción de tutela, la cual, no obstante su alta efectividad, no es aplicable en todos los casos debido a que:\label{socandres-18}

\begin{enumerate}[(A)]
\item Es subsidiaria, es decir, que sólo beneficia a aquellas personas que perciben subsidios estatales con el fin de protegerlas de su vulnerabilidad socioeconómica. 
\item Es específica, es decir, que sólo protege los derechos consagrados dentro del bloque de constitucionalidad.
\item Es subsidiaria, es decir, que sólo es aplicable cuando no existen otras vías de defensa judicial.
\item Es específica, es decir, que es única para la protección de los derechos consagrados en los diversos códigos legales.

\end{enumerate}
\item En el artículo 13 nuestra Carta Política reza lo siguiente: “Todas las personas nacen libres e iguales ante la ley, recibirán la misma protección y trato de las autoridades y gozarán de los mismos derechos, libertades y oportunidades sin ninguna discriminación por razones de sexo, raza, origen nacional o familiar, lengua, religión, opinión política o filosófica.
El Estado promoverá las condiciones para que la igualdad sea real y efectiva y adoptará medidas en favor de grupos discriminados o marginados.”\label{socandres-19}

Para la consecución de los objetivos inherentes a este artículo, ¿cuál de los siguientes planteamientos sería pertinente aplicar?


\begin{enumerate}[(A)]
\item Políticas que acompañen la defensa de postulados pluralistas tales como el etnocentrismo o la xenofobia para salvaguardar un ambiente de tolerancia a nivel nacional.
\item Campañas para rechazar el racismo, el altruismo y el fundamentalismo como representaciones de intolerancia y como manifestaciones explícitas de diversidad cultural.
\item Políticas que contemplen la aceptación de directrices y postulados tales como la segregación racial, en tanto que la Constitución Política propende en su artículo 16 el libre desarrollo de la personalidad (derechos conexos).
\item Campañas dirigidas al reconocimiento de la diferencia y el respeto de la diversidad, optando por la coexistencia pacífica entre las mayorías y minorías sociales, rechazando postulados tales como el racismo y la homofobia. 
\end{enumerate}

\subsubsection*{Responda las preguntas \ref{socandres-20} - \ref{socandres-25} con base en la siguiente información}

En el país Manchería, se presentan los resultados de las últimas 4 jornadas electorales discriminadas por año, partido político y porcentaje de votos de la siguiente manera: 

 \begin{tabular}{p{1cm}||p{6cm}}
Año & RESULTADO POR PARTIDO \\
\hline  \hline
\multirow{6}{*}{2000} & 1. Z con 48\% de votos (ganador). \\
 & 2. X con 34\% de votos.\\
 & 3. J con 10\% de votos.\\
& 4. K con 07\% de votos.\\
& * Voto en blanco: 01\% de votos.	\\
	& * Abstencionismo: 10\% del censo electoral.\\
	\hline
\end{tabular} 
\begin{tabular}{p{1cm}||p{6cm}}
\hline
\multirow{6}{*}{2004}& 1. X con 47\% de votos (ganador).\\
& 2.  Z con 30\% de votos.\\
& 3.  J con 10\% de votos.\\
& 4.  K con 08\% de votos.\\
& * Voto en blanco: 05\% de votos.\\
& * Abstencionismo: 12\% del censo electoral.\\ 
\hline
\end{tabular} 
\begin{tabular}{p{1cm}||p{6cm}}
\hline
\multirow{7}{*}{2008}& 1. X con 30 \% de votos (ganador).\\
			&		 2.  Z con 25 \% de votos.\\
			&		 3.  J con 10\% de votos.\\
			&		 4. K con 10\% de votos.\\
			&		 5. L con 05\% de votos.\\
			&		 * Voto en Blanco: 20\% de votos.\\
			&		 * Abstencionismo: 30\% del censo electoral.\\
			\hline
\end{tabular} 
\begin{tabular}{p{1cm}||p{6cm}}
\hline
\multirow{8}{*}{2012}&1. Z con 27\% de votos (ganador).\\
			&		  2.  X con 23\% de votos.\\
			&		  3.  L con 22 \% de votos.\\
			&		  4.  M con 10\% de votos.\\
			&		  5.  J con 03\%  de votos.\\
			&		  6.  K con 02\% de votos.\\
			&		  * Voto en blanco: 13\% de votos.\\
			&		  *Abstencionismo: 60\% del censo electoral \\
			\hline \hline
\end{tabular}  

\item De los anteriores resultados se puede concluir que:\label{socandres-20}

\begin{enumerate}[(A)]
\item El crecimiento inesperado del partido político L representa una amenaza para X y Z en las jornadas electorales representadas en la gráfica.
\item X y Z son los partidos políticos dominantes y por ello se puede afirmar que de acuerdo con las estadísticas, el próximo presidente será un candidato de cualquiera de estos dos partidos.
\item A pesar de que X y Z son los partidos más votados, empieza a existir una disconformidad política que se refleja en la disminución de votos para dichos partidos, el aumento progresivo del abstencionismo y el aumento de votos hacia nuevos partidos tales como L y M.
\item Para obtener la victoria sobre los partidos políticos dominantes, es necesaria una alianza entre los partidos L, M, J y K para las próximas elecciones en 2014.   

\end{enumerate}

\item Cuando una persona vota en blanco, quiere indicar:\label{socandres-21}

\begin{enumerate}[(A)]
\item Un profundo desconocimiento de las propuestas de los candidatos.
\item La indecisión en torno a votar por uno u otro candidato.
\item Una expresión de disentimiento o inconformidad.
\item La inoperancia de las formas políticas representativas electorales.

\end{enumerate}

\item Juan Pérez, ciudadano de Manchería, piensa que las elecciones del 2012 son ilegítimas. Uno de sus argumentos podría ser:\label{socandres-22}

\begin{enumerate}[(A)]
\item El candidato por el partido político X ganó las elecciones presidenciales sólo con el 27\% de los votos, lo cual no constituiría una mayoría electoral tomando como base el 100\% de la votación.
\item En dichas elecciones sólo el 40\% del censo electoral votó, por lo tanto, ningún candidato puede representar de manera alguna la voluntad de las mayorías.   
\item La voluntad de las personas que optaron por el voto en blanco, que en este caso fue del 13\% no fue tenida en cuenta y por lo tanto el candidato del partido político X no puede representar a la totalidad poblacional de Manchería.
\item No sufragó la totalidad de la población, por ejemplo la correspondiente a los niños, en ese caso las elecciones son ilegítimas por no permitir el voto de todos los habitantes de Manchería.

\end{enumerate}

\item Con la expresión ``censo electoral'' se hace referencia a:\label{socandres-23}\hrulefill\\
\_\hrulefill\\
\_\hrulefill\\
\_\hrulefill.

\item Los partidos políticos X y Z se han enfrentado históricamente por el dominio político de Manchería. Dicho enfrentamiento no sólo se ha llevado a cabo en el plano electoral, sino que ha atravesado diversas esferas tales como guerras y violencia en general que ha impedido a los habitantes de Manchería vivir en paz. Sin embargo, al analizar los resultados electorales de los últimos años X y Z observan que son los partidos políticos más votados. Así pues, para colocar fin a su enfrentamiento, los integrantes de estos partidos deciden crear un Frente Común con el objetivo de turnarse temporalmente el poder político, quedando configurado su acuerdo de la siguiente manera:\label{socandres-24}

En el año 2014 un candidato del partido X asumirá la presidencia.
En el año 2018 un candidato del partido Z asumirá la presidencia.
En el año 2022 un candidato del partido X asumirá la presidencia.
En el año 2026 un candidato del partido Z asumirá la presidencia.

Se podría afirmar que dicho acuerdo es:

\begin{enumerate}[(A)]
\item Justo, porque con dicho acuerdo finalizarían las guerras y la violencia que históricamente han tenido los partidos X y Z y de esta manera la población de Manchería podría vivir en paz.
\item Injusto, porque podrían haber sectores sociales y políticos que no se sientan representados por X ni por Z y que no encuentren alternativas para un ejercicio político democrático.
\item Justo, porque al ser los partidos políticos más votados, la voluntad de la mayoría se ve acogida y de esta manera se frenaría la violencia que existe en Manchería. 
\item Injusto, porque al turnarse el poder X y Z no habría una continuidad política que permitiera definir el rumbo social, político y económico de Manchería, lo cual llenaría de incertidumbre a sus habitantes.

\end{enumerate}

\item Una situación análoga a dicho acuerdo es:\label{socandres-25}

\begin{enumerate}[(A)]
\item Frente Nacional de Colombia.
\item La Perestroika de la URSS.
\item Proceso de Apartheid en África.
\item Repartición de África por parte de potencias imperiales en la conferencia de Berlín 1884-1885.
\end{enumerate}


%%%%%%%%%%%%%%%%%%%%%%%%%%%%%
\end{enumerate}
%%%%%%%%%%%%%%%%%%%%%%%%%%%%%5


