\begin{itemize}

%%%%%%%%%%%%%%%%%%%%%%%%%%%%%%%%%%%%%%%%%%%%%%%%%%%%%%%%%%%%%%%%%%%%%%%%%%

\item[P. \ref{vic-5}] Lo primero que se debe realizar es unificar las unidades de tiempo y tomar la más pequeña que en este caso es el segundo.
5/3 de minuto (1 minuto es igual a 60 segundos) = $\frac{5}{3	}x60=100$ segundos.

4 minutos y 15 segundos = 60 x 4 = 240 + 15 = 255 segundos.
 1/12 de hora (1 Hora es igual a 3600 segundos) $\frac{1}{12}x3600=300$ segundos.
3 y 3/4 de minutos (3 minutos = 180 segundos y 3/4 de minuto = 3/4x60 = 45 segundos)  total 225 segundos

Entonces para una responder una pregunta se demora 100 + 255 + 300 + 225 = 880 segundos.

Carlos se demora por cada pregunta 880 segundos y si son 35 preguntas se multiplican estos dos valores 
880 x 35  = 30.800 segundos.

%%%%%%%%%%%%%%%%%%%%%%%%%%%%%%%%%%%%%%%%%%%%%%%%%%%%%%%%%%%%%%%%%%%%%%%%%%

\item[P. \ref{vic-12}] Empleando un diagrama de  Venn  representamos por $B$ el conjunto de los artistas que bailan, por  $C$  el conjunto de los artistas de cantan y por $U$ el conjunto de todos los artistas; de esta manera, en la intersección de los conjuntos $B\cap C$ irían los 12 artistas que bailan y cantan, como los artistas que bailan son 16 y 12 están en la intersección, los artistas que solo bailan serían 4; de igual forma como los artistas que cantan son 25 y 12 están en la intersección, 13 serían los artistas que solo cantan, de tal forma que 3 serían los artistas que ni cantan ni bailan para poder así completar el conjunto $U$ de los 32 artistas.

%%%%%%%%%%%%%%%%%%%%%%%%%%%%%%%%%%%%%%%%%%%%%%%%%%%%%%%%%%%%%%%%%%%%%%%%%%
%%%%%%%%%%% Diana %%%%%%%%%%%%%%%%%%
%%%%%%%%%%%%%%%%%%%%%%%%%%%%%%%%%%%%%%%%%%%%%%%%%%%%%%%%%%%%%%%%%%%%%%%%%%

\item[P. \ref{dia-10}] Cuando una esfera de plomo y una de madera de igual radio caen, la resistencia del aire $R$ que actúa sobre cada una de ellas es prácticamente igual. Que se puede decir sobre las aceleraciones de las esferas?\\ \\
Para ambas esferas es cierto que $mg-R=ma$ por lo tanto $a=g-\frac{R}{m}$, es decir depende inversamente proporcional a la masa de la esfera. En esferas del mismo volumen de plomo y de madera $m_{Plomo}>m_{Madera}$, por lo tanto la aceleración de la esfera de plomo será mayor que la de madera.

%%%%%%%%%%%%%%%%%%%%%%%%%%%%%%%%%%%%%%%%%%%%%%%%%%%%%%%%%%%%%%%%%%%%%%%%%%

\item[P. \ref{dia-11}] En una campaña ecológica realizada se ha notado que si se vierte una gota de aceite de volumen $0.01cm^3$ sobre una piscina esta se expande uniformemente de tal manera que dicha capa tiene un espesor de $10^{-9}m$. Cuál es el área de la piscina de agua contaminada?\\
El volumen de aceite no puede cambiar así este sobre la superficie de la piscina, asi que;
$V_i=\epsilon \times A$, tomando $\epsilon=10^{-9}m$ y $V_i=0.01cm^3=0.01\times10^{-6}m^3$, luego;\\
\begin{center}
$A=\frac{0.01\times10^{-6}m^3}{10^{-9}m}=10m^2$
\end{center}

%%%%%%%%%%%%%%%%%%%%%%%%%%%%%%%%%%%%%%%%%%%%%%%%%%%%%%%%%%%%%%%%%%%%%%%%%%


\item[P. \ref{dia-13}] Porque los bomberos tienen que sujetar fuertemente la manguera cuando se lanza agua a alta presión para apagar un incendio?\\ \\ La cantidad de movimiento inicial es cero, y debe conservarse por lo tanto la manguera intentará empujar hacia atrás al bombero y este a su vez debería mantenerse en pie.

%%%%%%%%%%%%%%%%%%%%%%%%%%%%%%%%%%%%%%%%%%%%%%%%%%%%%%%%%%%%%%%%%%%%%%%%%%


\item[P. \ref{dia-14}] Un tubo de crema dental se cierra en la fábrica a la altura de mar y que se lleva un cargamento de estas a Bogotá para su comercialización. Cuando usted compre uno de estos tubos y lo abra la crema dental:\\
A la altura del mar la presión será mayor que la presión en Bogotá puesto que Bogotá es mas alto (menor presión). Si empacan la crema dental a presión mayor cuando la lleven a un lugar con menor presión atmosférica, al abrirla debido a la diferencia de presiones esta saldrá. Esto es, la presión es mayor dentro del tubo que fuera de él.


%%%%%%%%%%%%%%%%%%%%%%%%%%%%%%%%%%%%%%%%%%%%%%%%%%%%%%%%%%%%%%%%%%%%%%%%%%

\item[P. \ref{dia-22}] En un juego se tiene que mover una pelota pesada desde el reposo hasta una distancia máxima, uno de los jugadores tira la pelota de $5$ Kg a $25$ m en $5$ s. Asuma que la aceleración es constante, Cuál es la fuerza horizontal que el jugador ejerció sobre la pelota?\\
Se tiene que $d=\frac{at^2}{2}+V_{0}t$, en este caso $V_0=0$, esto es $25m=a\frac{5^2 s^2}{2}$, despejando para $a$ se encuentra que $a=\frac{2*25m}{5^2 s^2}=2\frac{m}{s^2}$, y teniendo que $F=ma$ entonces $F=5Kg*2\frac{m}{s^2}=10N$



\end{itemize}

