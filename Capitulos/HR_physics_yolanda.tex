\noindent \ref{yolf-1} A\\
\ref{yolf-2} C\\
\ref{yolf-3} B\\
\ref{yolf-4} A\\
\ref{yolf-5} B\\
\ref{yolf-6} C\\
\ref{yolf-7} D\\
\ref{yolf-8} B\\
\ref{yolf-9} C\\
\ref{yolf-10} B\\
\ref{yolf-11} D\\
\ref{yolf-12} D\\
\ref{yolf-13} El producto de la fuerza neta paralela al plano de aplicación por la distancia en la cual actúa dicha fuerza.\\
\ref{yolf-14} C\\
\ref{yolf-15} Es la relación entre el trabajo que hace una fuerza y el tiempo que dura actuando dicha fuerza.\\
\ref{yolf-16} B\\
\ref{yolf-17} D\\
\ref{yolf-18} B\\
\ref{yolf-19} C\\
\ref{yolf-20} Estudia los fenómenos relativos al calor y la temperatura y los cambios experimentados en la materia cuando ellos actúan.\\
\ref{yolf-21} B\\
\ref{yolf-22} D\\
\ref{yolf-23} Propiedad de los cuerpos de no modificar su estado de reposo o movimiento.\\
\ref{yolf-24} D\\
\ref{yolf-25} La diferencia de potencial entre dos puntos o la diferencia de energía potencial eléctrica por unidad de carga entre dos puntos.\\



