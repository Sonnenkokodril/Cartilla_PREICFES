\subsubsection*{Las preguntas del \ref{yolma-1} al \ref{yolma-3} se contestan con base en la siguiente información}


Un globo circular comienza a inflarse en un instante $t=0$; el radio de dicho globo en función del tiempo n, viene dado por la expresión $r (t)=2t+1$, donde $t$ representa el tiempo transcurrido en segundos y $r$ se mide en centímetros. Además debido a la calidad del caucho usado en la elaboración del globo, este explota cuando el diámetro del globo  alcanza 30 centímetros.\\
\begin{enumerate}

\item Con base en la información suministrada es válido afirmar que:\label{yolma-1}\\



\begin{enumerate}[(A)]
\item  En el instante en que empieza a ser llenado, el radio del globo es 2 centímetros
\item Por cada segundo transcurrido el volumen del globo aumenta en 2 centímetros cúbicos
\item El radio del globo y el tiempo transcurrido se comportan como magnitudes directamente proporcionales
\item Al momento de estallar el radio del globo es de 15 centímetros 
\end{enumerate}

%%%%%%%%%%%%%%%%%%%%%%%%%%%%%
\newpage
\item  El tiempo que transcurre desde el inicio del proceso de inflado hasta el instante en el cual el globo estalla es  \label{yolma-2}\\

\begin{enumerate}[(A)]
\item  30 segundos
\item 21 segundos
\item 15 segundos
\item 7 segundos
\end{enumerate}

%%%%%%%%%%%%%%%%%%%%%%%%%%%%%

\item  La expresión que permite determinar el volumen del globo, en función del tiempo transcurrido desde el inicio del proceso de inflado es \label{yolma-3}\\

\begin{enumerate}[(A)]
\item  Cuatro tercios de pi por radio al cubo
\item Cuatro tercios del radio al cubo
\item Cuatro veces el radio al cubo
\item Cuatro tercios de pi por el radio
\end{enumerate}

%%%%%%%%%%%%%%%%%%%%%%%%%%%%%
\subsubsection*{Responda las preguntas \ref{yolma-4} y \ref{yolma-5} de acuerdo a la siguiente información}

Los siguientes datos indica los valores obtenidos para dos funciones $f$ y $g$ al ser evaluados en $x=0$.  $f(0)=3$; $f’(0)=1$; $g(0)=2$; $g’(0)=1$


\newpage
\item Con base en la información suministrada es correcto afirmar que: \label{yolma-4}\\

\begin{enumerate}[(A)]
\item  La función $f$ no es diferenciable en $x=0$
\item Las funciones $f$ y $g$ son continuas en $x=0$
\item La función $g$ es diferenciable en $x=0$ pero no es continua
\item Tanto $f$ como $g$ no son diferenciables en $x=0$
\end{enumerate}

%%%%%%%%%%%%%%%%%%%%%%%%%%%%%

\item De acuerdo con la información suministrada se verifica que la función $f*g$ es diferenciable en $x=0$ y además, que $(f*g)’(0)$ es igual a \label{yolma-5}\\

\begin{enumerate}[(A)]
\item  5
\item 1
\item 6
\item 2
\end{enumerate}

%%%%%%%%%%%%%%%%%%%%%%%%%%%%%

\subsubsection*{Responda las preguntas \ref{yolma-6}  al \ref{yolma-9} con base en la siguiente información}

\noindent En el proceso de elaboración de un determinado materia, se estipula que dada una cantidad $M$, en kilogramos, de masa a procesar para obtener el material, la temperatura que alcanza el proceso viene dada por la expresión: 

\begin{equation*}
M^2+20M, \text{en grados centígrados}
\end{equation*}

Por razones de seguridad, la temperatura del proceso, no debe superar los 12.000 grados centígrados.

\newpage
\item La información dada en el texto, en términos de inecuaciones se expresa como:  \label{yolma-6}\\

\begin{enumerate}[(A)]
\item  $M^2+12.000\leq 20M$
\item $12.000\leq M^2+20M$
\item $12.000+20M\leq M^2$
\item $M^2+20M\leq 12.000$
\end{enumerate}

%%%%%%%%%%%%%%%%%%%%%%%%%%%%%

\item Con base en la información, se puede procesar 40Kg de masa. En tal caso, la temperatura que alcanza el proceso es:\label{yolma-7}\\

\begin{enumerate}[(A)]
\item  1.500 grados centígrados
\item 1.800 grados centígrados
\item 2.100 grados centígrados
\item 2.400 grados centígrados
\end{enumerate}

%%%%%%%%%%%%%%%%%%%%%%%%%%%%%

\item  Con base en la expresión que determina la temperatura, según la cantidad de materia procesada, se observa que a mayor cantidad de materia procesada,  la temperatura \label{yolma-8}\\

\begin{enumerate}[(A)]
\item  Aumenta directamente proporcional a la masa
\item Aumenta linealmente proporcional a la masa
\item Aumenta cuadráticamente proporcional a la masa
\item Aumenta en algunos casos y en otros no varia
\end{enumerate}

%%%%%%%%%%%%%%%%%%%%%%%%%%%%%

\newpage

\item Una información adicional, establece que con el objeto de reducir costos de producción, la temperatura mínima del proceso de producción debe ser 1.500 grados; por lo tanto para el proceso  la cantidad de materia para el proceso debe \label{yolma-9}\\

\begin{enumerate}[(A)]
\item  Menor a 50Kg porque así se economiza mas energía
\item Exactamente 50Kg porque es la cantidad de materia que permite que la temperatura sea 1.500 grados
\item Algo más de 50Kg porque la graduación de la maquina no es exacta
\item Mucho más de 50Kg porque los niveles de producción así lo exigen
\end{enumerate}

%%%%%%%%%%%%%%%%%%%%%%%%%%%%%

\subsubsection*{Responda las preguntas \ref{yolma-10} al \ref{yolma-12} de acuerdo a la siguiente información}

\begin{center}
\begin{tabular}{c|ccc}
\hline 
\hline 
 & Efectivo & Cheque & Tarjeta \\ 
\hline 
\hline 
Hombre & 28 & 60 & 37 \\ 
Mujer & 40 & 39 & 46 \\ 
\hline 
\hline 
\end{tabular} 
\end{center}
\item  De la información en la tabla es válido afirmar que: \label{yolma-10}\\

\begin{enumerate}[(A)]
\item  El número de mujeres encuestadas es superior al de hombres encuestados
\item La cantidad de personas encuestadas es desconocida
\item 68 de los encuestados pagaron sus compras en efectivo
\item El número de mujeres encuestadas es inferior a 110
\end{enumerate}

%%%%%%%%%%%%%%%%%%%%%%%%%%%%%
\newpage
\item De acuerdo con la tabla se infiere que la probabilidad de que una persona que realice una compra en el supermercado, sea mujer y cancele en efectivo es: \label{yolma-11}\\

\begin{enumerate}[(A)]
\item  40/125
\item 40/250
\item 28/40
\item 85/125
\end{enumerate}

%%%%%%%%%%%%%%%%%%%%%%%%%%%%%

\item La probabilidad de que un hombre pague con cheque, es mayor que la probabilidad que: \label{yolma-12}\\

\begin{enumerate}[(A)]
\item  Pague en efectivo o con tarjeta
\item Una mujer pague en efectivo o en cheque
\item Una mujer pague en efectivo o con tarjeta
\item Una mujer pague en cheque
\end{enumerate}

%%%%%%%%%%%%%%%%%%%%%%%%%%%%%

\item Los divisores de un numero son:\label{yolma-13}\\\hrulefill\\
\_\hrulefill\\
\_\hrulefill\\



%%%%%%%%%%%%%%%%%%%%%%%%%%%%%

\subsubsection*{Responda las preguntas \ref{yolma-14} al \ref{yolma-16} de acuerdo con la siguiente información}
Para una terna de ángulos $A$, $B$, y $C$ se verifican las siguientes relaciones:

\begin{gather*}
\text{Sen}(A)=\frac13; \text{Sen}(B)=\frac35; \text{Sen}(C)=\frac{5}{13}\\
\text{Cos}(A)=\frac{2\sqrt{2}}{3}; \text{Cos}(B)=\frac45; \text{Cos}(C)=\frac{12}{13}
\end{gather*}

Los tres ángulos son agudos

\newpage
\item  Con base en la información suministrada, para determinar el valor de Tan($B$), es necesario\label{yolma-14}\\

\begin{enumerate}[(A)]
\item  Multiplicar $\frac35$ por $\frac45$
\item Dividir $\frac45$ entre $\frac35$
\item Sumar $\frac35$ y $\frac45$
\item Dividir $\frac35$ entre $\frac45$
\end{enumerate}

%%%%%%%%%%%%%%%%%%%%%%%%%%%%%

\item A partir de la información suministrada, es posible determinar que el valor $\frac{12}{5}$ es equivalente a  \label{yolma-15}\\
\begin{enumerate}[(A)]
\item  Tan($A$)
\item Cot($B$)
\item Sec($C$)
\item Ninguna de las anteriores
\end{enumerate}

%%%%%%%%%%%%%%%%%%%%%%%%%%%%%
\item  Con base en la información suministrada, para determinar el valor de Sen($A+B$), es necesario determinar el resultado de: \label{yolma-16}
\begin{enumerate}[(A)]
\item  Sen($A$)+Sen($B$)
\item Sen($A$)Cos($B$)+Sen($B$)Cos($A$)
\item Sen($A$)Cos($B$)-Sen($B$)Cos($A$)
\item Sen($A$)-Sen($B$)
\end{enumerate}

%%%%%%%%%%%%%%%%%%%%%%%%%%%%%

\item  Las relaciones trigonométricas se establecen\label{yolma-17}\hrulefill\\
\_\hrulefill\\
\_\hrulefill
\_\hrulefill.


%%%%%%%%%%%%%%%%%%%%%%%%%%%%%

\item Si un cilindro y un cono de igual altura y radio se acoplan por sus bases es incorrecto afirmar que: \label{yolma-18}
\begin{enumerate}[(A)]
\item  El volumen total de la pieza es $\frac43$ del volumen del cilindro
\item El volumen del cilindro es tres veces el volumen del cono
\item El volumen del cono es tres veces el volumen del cilindro
\item La altura total de la pieza es el doble de la del cono
\end{enumerate}

%%%%%%%%%%%%%%%%%%%%%%%%%%%%%

\item El volumen es: \label{yolma-19}\hrulefill\\
\_\hrulefill\\
\_\hrulefill
\_\hrulefill.

%%%%%%%%%%%%%%%%%%%%%%%%%%%%%

\subsubsection*{Responda las preguntas \ref{yolma-20} al \ref{yolma-22} de acuerdo con la siguiente información}

Una empresa promotora de salud ofrece dos planes de afiliación: el primero, tiene un valor de \$800.000 anuales y el segundo la mitad del valor del primero, más \$12.000 por cada día de hospitalización.

\item Para una persona que dure hospitalizada mes y medio, tiene mayor beneficio económico con: \label{yolma-20}\\

\begin{enumerate}[(A)]
\item  El segundo plan porque paga menos dinero
\item El primer plan
\item Cualquiera de los planes le sirve
\item El segundo plan si el valor es una fracción del primero
\end{enumerate}

%%%%%%%%%%%%%%%%%%%%%%%%%%%%%

\item Si la fracción que determina el valor del segundo plan se cuadriplicara, el tiempo que podría permanecer una persona hospitalizada para obtener el mismo costo del primero sería: \label{yolma-21}\\

\begin{enumerate}[(A)]
\item  Menos de un mes
\item Un mes exactamente
\item Más de un mes
\item Hasta un año
\end{enumerate}

%%%%%%%%%%%%%%%%%%%%%%%%%%%%%
\newpage
\item Si la fracción que determina el valor del segundo plan disminuye su denominador que pasaría con respecto al primer plan durante dos meses de hospitalización: \label{yolma-22}\\

\begin{enumerate}[(A)]
\item  El día de hospitalización resultaría 50\% más costoso respecto del primer plan
\item El día de hospitalización resultaría igual de costoso que en el primero
\item El día de hospitalización seria casi el doble de costoso que en el primero
\item El día de hospitalización seria el triple de costoso del primero
\end{enumerate}

%%%%%%%%%%%%%%%%%%%%%%%%%%%%%

\item La razón de dos segmentos es: \label{yolma-23}\hrulefill\\
\_\hrulefill\\
\_\hrulefill
\_\hrulefill.


%%%%%%%%%%%%%%%%%%%%%%%%%%%%%
\newpage
\item La compañía Personal Teaching ofrece dos tipos de cursos en áreas no académicas diferentes. Para la semana entrante dispone de 60 horas de trabajo destinadas a dictar los dos cursos. Además como dichos cursos son los productos bandera de la compañía, a la gerencia le interesa utilizar las 60 horas disponibles de esa semana. El curso de cocina requiere 1.5 horas de trabajo para su montaje y el de música requiere 1.25 horas. La ecuación que relaciona la cantidad de cursos que se ofrecerán de cada uno con el total de horas disponibles es: \label{yolma-24}
\begin{enumerate}[(A)]
\item  $1.5C+1.25M=60$
\item $1.5C-1.25M=60$
\item $1.25M-1.25C=60$
\item $1.5C+1.5M=60$
\end{enumerate}

%%%%%%%%%%%%%%%%%%%%%%%%%%%%%

\item La trigonometría es: \label{yolma-25}\hrulefill\\
\_\hrulefill\\
\_\hrulefill
\_\hrulefill.

%%%%%%%%%%%%%%%%%%%%%%%%%%%%%
\end{enumerate}
%%%%%%%%%%%%%%%%%%%%%%%%%%%%%5

