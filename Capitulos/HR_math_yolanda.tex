
%%%%%%%%%%%%%%%%%%%%%%%%%%%%%%%%%%%%%%%%%%%%
%%%%%%%%%%%%%%%%  RESPUESTAS %%%%%%%%%%%%%%
%%%%%%%%%%%%%%%%%%%%%%%%%%%%%%%%%%%%%%%%%%
\noindent \ref{yolma-1} D\\
\ref{yolma-2} D\\
\ref{yolma-3} A\\
\ref{yolma-4} B\\
\ref{yolma-5} A\\
\ref{yolma-6} D\
\ref{yolma-7} D\\
\ref{yolma-8} C\\
\ref{yolma-9} B\\
\ref{yolma-10} C\\
\ref{yolma-11} B\\
\ref{yolma-12} A\\
\ref{yolma-13} Los números que dividen exactamente a otro.\\
\ref{yolma-14} D\\
\ref{yolma-15} D\\
\ref{yolma-16} D\\
\ref{yolma-17} Entre los lados de un triángulo rectángulo.\\
\ref{yolma-18} C\\
\ref{yolma-19} Cantidad de espacio ocupado por un cuerpo.\\
\ref{yolma-20} B\\
\ref{yolma-21} A\\
\ref{yolma-22} C\\
\ref{yolma-23} Es la relación en que están los números que expresan sus longitudes-\\
\ref{yolma-24} A\\
\ref{yolma-25} Parte de las matemáticas que estudia la solución de triángulos, dando normas para calcular lados y ángulos desconocidos a partir de datos conocidos.\\


