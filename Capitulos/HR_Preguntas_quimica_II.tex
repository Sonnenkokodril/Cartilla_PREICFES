

%%%%%%%%%%%%%%%%%%%%%%%%%%%%%%%%%%%%%%%%%%%%
%%%%%%%%%%%%%%%%  RESPUESTAS %%%%%%%%%%%%%%
%%%%%%%%%%%%%%%%%%%%%%%%%%%%%%%%%%%%%%%%%%
\noindent \ref{jenn-1} B\\
\ref{jenn-2} D\\
\ref{jenn-3} ?\\
\ref{jenn-4} A\\
\ref{jenn-5} B\\
\ref{jenn-6} C\\
\ref{jenn-7} ???\\
\ref{jenn-8} ???\\
\ref{jenn-9} A\\\
\ref{jenn-10} A\\
\ref{jenn-11} A\\
\ref{jenn-12} A\\
\ref{jenn-13} B\\
\ref{jenn-14} A\\
\ref{jenn-15} C\\
\ref{jenn-16} A\\
\ref{jenn-17} D\\
\ref{jenn-18} C\\
\ref{jenn-19} ???\\
\ref{jenn-20} ???\\
\ref{jenn-21} Porque esta sustancia adsorbe el yodo el cual se introduce entre las hélices que forma  la molécula de almidón produciendo una coloración azul intensa.\\
\ref{jenn-22} El tamaño de las partículas son más grandes en un coloide, haciendo que se vean las dos fases; mientras en una solución homogénea solo se ve una fase.\\
\ref{jenn-23} ???\\
\ref{jenn-24} Acido 2-metil-hexanoico\\
\ref{jenn-25} El reactivo limite condiciona la cantidad de sustancia que produce, debido a que este es el primero que se consume durante una reacción.\\


