\noindent \ref{bio-1} C\\
\ref{bio-2} C\\
\ref{bio-3} A\\
\ref{bio-4} A\\
\ref{bio-5} C\\
\ref{bio-6} C\\
\ref{bio-7} C\\
\ref{bio-8} D\\
\ref{bio-9} B\\
\ref{bio-10} D\\
\ref{bio-11} B\\
\ref{bio-12} C\\
\ref{bio-13} B\\
\ref{bio-14} B\\
\ref{bio-15} A\\
\ref{bio-16} B\\
\ref{bio-17} A\\
\ref{bio-18} B\\
\ref{bio-19} C\\
\ref{bio-20} A\\
\ref{bio-21} A\\
\ref{bio-22} C\\
\ref{bio-23} C\\
\ref{bio-24} C\\
\ref{bio-25} 32. Al extinguirse los peces pequeños la población de sapos permanece igual, al observar las redes alimenticias estos  no son alimento directo del sapo, esté se alimenta de moscas y libélulas, por el contrario  los peces alimentan  a depredadores más grandes, lo posible es que si se acaban los peces pequeños estos  depredadores remplazaran  estos organismos por otro tipo de alimento organismo en el caso de la nutria esta se alimenta de tres animales  diferentes.\\
\ref{bio-26} D\\
\ref{bio-27} C\\



