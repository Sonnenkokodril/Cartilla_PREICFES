%%% Hoja de respuestas de Ciencias Sociales III
\noindent \ref{sociii-1}. C\\
\ref{sociii-2}. B\\
\ref{sociii-3}. C\\
\ref{sociii-4}. 3. Son derechos inherentes a todos los seres humanos, sin distinción alguna de nacionalidad, lugar de residencia, sexo, origen nacional o étnico, color, religión, lengua, o cualquier otra condición. \\ 
\ref{sociii-5}. C \\
\ref{sociii-6}. C\\
\ref{sociii-7}. B\\
\ref{sociii-8}. A\\
\ref{sociii-9}. D \\ 
\ref{sociii-10}. D\\
\ref{sociii-11}. Camilo Torres Restrepo.\\
\ref{sociii-12}. B\\
\ref{sociii-13}. Regular la manera en que se presentan los conflictos armados con el fin de proteger a las personas que no participan en las hostilidades y a los que ya no pueden seguir participando en los combates.\\
\ref{sociii-14}. D\\
\ref{sociii-15}. D\\
\ref{sociii-16}. B\\
\ref{sociii-17}. C\\
\ref{sociii-18}. Que se llegue a un acuerdo entre las guerrillas y el gobierno no quiere decir que el conflicto va a desaparecer, pues la insurgencia armada es sólo una expresión más del conflicto y por tanto no puede ser considerada como causa sino como consecuencia del mismo.\\
\ref{sociii-19}. A\\
\ref{sociii-20}. A\\
\ref{sociii-21}. B\\
\ref{sociii-22}. D\\
\ref{sociii-23}. B\\
\ref{sociii-24}. B\\
\ref{sociii-25}. D\\
